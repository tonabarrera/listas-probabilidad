\documentclass[11pt,a4paper]{article}
\usepackage[utf8]{inputenc}
\usepackage[spanish]{babel}
\usepackage{amsmath}
\usepackage{amsfonts}
\usepackage{amssymb}
\usepackage{graphicx}
\usepackage[left=2cm,right=2cm,top=2cm,bottom=2cm]{geometry}
\author{Jose Ricardo Lopez Garcia}
\title{Salvando el semestre :'v}
\begin{document}
\maketitle

\subsection{Lista 3}

\begin{enumerate}
\setcounter{enumi}{63}
 \item Se sabe que el gatorade es eficiente en el 72\% de los casos en los que se usa para aliviar los efectos del laboratorio del día anterior. \\
 Se ha desarrollado un nuevo sabor y las pruebas demuestra que fue efectivo en 42 de los 50 casos.\\
¿Es esta una evidencia suficientemente fuerte paraprobar que el nuevo sabor es mas efectivo que el viejo con una confianza del 91\%?
	\\\textbf{Solución}
	
	\begin{gather*}	 
	n_{1} = 50\\
	n_{2} = 50\\
	\overline{x_{1}} = 36\\
	\overline{x_{2}} = 42\\	
	\sigma_{1}^{2} = npq =11.52\\
	\sigma_{2}^{2} = npq =12.58\\
	z = \frac{\overline{x_{1}} - \overline{x_{2}}}{\sqrt{\frac{\sigma_{1}^{2}}{n_{1}^{2}} + \frac{\sigma_{2}^{2}}{n_{2}^{2}}}}\\= \frac{36-42}{\sqrt{\frac{11.52}{50^{2}}}+\frac{12.18}{50^{2}}}\\
	=\frac{-6}{0.097} = -61.85\\
	\end{gather*}
	\begin{center}
	$\therefore$ si es prueba suficiente
	\end{center}
\item Suponga que un dispositivo determinado contiene cinco circuitos electrónicos; se supone que el tiempo(en horas) hasta que falle cada uno de los siguientes circuitos es una variable exponencial con media igual a 1000 y que el dispositivo trabaja solamente mientras trabajan los 5 circuitos.\\
¿Cuál es la probabilidad de que el dispositivo trabaje al menos 100 horas?
	\\\textbf{Solución}
	
	\begin{gather*}	 
	\mu = 1000 = np = \lambda\\
	X = 100\\
	P(X \geqslant 100) = \int_{X=100}^{X=\infty} (\lambda)(e^{-\lambda X})dX - \int_{X = 0}^{X = 100} (\lambda)(e^{-\lambda X})dX\\
	= 1 - e^{-\mu X}|_{\infty}^{100}\\
	=1 + 0 - 1 = 0
	\end{gather*}\\
\item Supóngase que la cantidad real de café colocada por una maquina llena de frascos de "n" onzas es una variable aleatoria con distribución normal con varianza de 0.0025 de onza.\\
Si solo el 3\% de los frascos van a contener menos de n onzas de café.\\
¿Cuál debe de ser el contenido medio de estos frascos?.\\
	\\\textbf{Solución}
	
	\begin{gather*}	 
	 Z_{\frac{100 - 97}{2}} = Z_{.015} = -2.17\\
	 Z = \frac{n -\mu}{\sigma}
	 \mu = n - Z\sigma = n + .0054
	\end{gather*}\\
\item Supóngase que la cantidad real de pintura colocada por una maquina llena de latas de ocho galones es una variable aleatoria con distribución normal con desviación estándar 0.0025 de onza.\\
Si solo el 3\% de las latas van a contener menos de 3 galones de pintura.\\
¿Cuál debe de ser el contenido medio de estas latas?.\\
	\\\textbf{Solución}
	
	\begin{gather*}	 
	 Z_{\frac{100 - 97}{2}} = Z_{.015} = -2.17\\
	 Z = \frac{X -\mu}{\sigma}
	 \mu = X - Z\sigma = 8.0054
	\end{gather*}\\
\item El departamento de seguridad de una fabrica desea saber si el tiempo promedio real que requiere el velador para realizar una ronda nocturna es de 30 min.\\
Si en una muestra tomada al azar requiere de 32 rondas, el velador promedio 30.8 min con una desviación estándar de 1.5 min.\\
Determine con un nivel de confianza del 99\% si es evidencia suficiente para rechazar la hipotesis nula $\mu$ = 30 a favor de la hipotesis alternativa de $\mu$ $\neq$ 30
	\\\textbf{Solución}
	\begin{gather*}
		 n	= 32\\
		 \mu = 30.8\\
		 \sigma = 1.5\\
		 \alpha = 99\% = .99
	\end{gather*}
	\begin{gather*}	 
	 Z_{\frac{100 - 99}{2}} = Z_{.005} = -2.57\\
	 Z_{\frac{100 + 99}{2}} = Z_{.995} = 2.57\\
	\end{gather*}\\
	\begin{gather*}	 
	 z = \frac{\mu - \mu^{'}}{\frac{\sigma}{\sqrt{n}}} = \frac{30.8 - 30}{\frac{1.5}{\sqrt{32}}} = 3.016
	\end{gather*}\\
	\begin{center}
	$\therefore$ Se rechaza la hipotesis nula $\mu$ = 30•
	\end{center}
	 
	
\item El fabricante de un producto removedor de manchas afirma que su producto remueve cuando menos en 90\% de todas las manchas.\\
¿Qué podemos concluir acerca de estas afirmaciones en un  95\% si el producto solo alimino 174 de 200 manchas elegidas al azar de ropa manchada?
	\\\textbf{Solución}
	\begin{gather*}
		\alpha = 95\%\\
		 n	= 200\\
		 p = \frac{174}{200} = .87\\
		 q = \frac{26}{200} = .13\\
	\end{gather*}
	\begin{gather*}
		\mu = np = (200)(.87) = 174\\
		 \sigma = \sqrt{npq} = \sqrt{(200)(.87)(.13)} = 4.756\\
	\end{gather*}
	\begin{gather*}	 
	 Z_{\frac{100 - 95}{2}} = Z_{.05} = -1.64\\
	 Z_{\frac{100 + 95}{2}} = Z_{.95} = 1.64\\
	\end{gather*}\\
	\begin{gather*}	 
	 (174 - 1.64(\frac{4.756}{\sqrt{200}}) \leqslant \mu \leqslant 174 + 1.64(\frac{4.756}{\sqrt{200}}))\\
	 (173.44 \leqslant \mu \leqslant 174.5528)\\
	 \therefore (173,175)\\
	\end{gather*}\\
	\begin{gather*}	 
	 \alpha = 95\%\\
		 n	= 200\\
		 p = \frac{174}{200} = .9\\
		 q = \frac{26}{200} = .1\\
		 \mu = 180\\
		 \sigma = 4.24\\
		 (179.5087 \leqslant \mu \leqslant 180.4913)\\
		 \therefore (179,180)\\
	\end{gather*}\\
	\begin{center}
	$\therefore$ no se cumple el producto\\
	\end{center}
	
	\item Durante varios años, se había aplicado un examen diagnostico a todos los alumnos de tercer semestre de la ESCOM. Si 64 estudiantes seleccionados al azar tardaron en promedio 28.5 minutos en resolver el examen con una varianza de 9.3.\\
¿Cuánto se esperaría que tardaran entre 27 y 32 minutos en resolver el examen?
	\\\textbf{Solución}
	\begin{gather*}
		 n	= 64\\
		 \mu = 28.5\\		 
	\end{gather*}
	\begin{gather*}
		 \sigma = \sqrt{\sigma^{2}} = \sqrt{9.3} = 3.0496\\
	\end{gather*}
	\begin{gather*}	 
	 Z_{min} = \frac{32 - 28.5}{3.0496} = 1.147\\
	 Z_{max} = \frac{27 - 28.5}{3.0496} = .4918\\
	 P(x) = P(1.147) - P(.4918)\\ = .872856848 - .3120669484\\ = .5607898996 = 56.07\%
	\end{gather*}\\
	\item Si el 23\% de todos los pacientes con presión sanguínea elevada tienen efectos colaterales nocivos por la ingesta de cierto medicamento.\\
Utilice la aproximación normal para obtener la probabilidad de que entre 12n de estos pacientes tratados con este medicamento unos 3n presentaran efectos colaterales nocivos
	\\\textbf{Solución}
	\begin{gather*}
		"n" = 5\\
		 n	= (12)(5) = 60\\
		 Exitos : (3)(5) = 15\\
		 p = .23\\
		 q = .77\\		 
	\end{gather*}
	\begin{gather*}
		\mu = np = (60)(.23) = 13.8\\
		 \sigma = \sqrt{npq} = \sqrt{(60)(.23)(.77)} = 3.856943648 = 3.857\\
	\end{gather*}
	\begin{gather*}	 
	 14.5 \leqslant x \leqslant 15.5\\	
	 Z_{min} = \frac{14.5 - 13.8}{3.857} = .1814\\
	 Z_{max} = \frac{15.5 - 13.8}{3.857} = .4407\\
	 P(14.5 \leqslant x \leqslant 15.5) =  P(.4407) - P(.1814)\\ = .67003144 - .57142371 = .0986077 = 9.86\%
	\end{gather*}
	
\end{enumerate}
\end{document}