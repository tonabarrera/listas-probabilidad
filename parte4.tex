%esta es la parte 4 paps, del 22 - 28 que corresponde a los problemas  22-26 de la lista 1 y a los 1-2 de la 2

\begin{enumerate}
	
\end{enumerate}

%el problema del jefe ,seria el 39 que es el 13 de la lista 2
13. Un profesor de ESCOM selecciona al azar a 3 alumnos de un grupo de 10 para aprobarlos. Suponiendo que el semestre anterior aprobó cuatro de esos 10 alumnos, determine la probabilidad de que exactamente 2 de los 3 alumnos hayan aprobado en el semestre anterior.
	\\\textbf{Solución}
	\\\text{La variable tiene una distribución hipergeométrica} \\
	\begin{gather*}
		N = 10 \\
		n = 3 \\
		k = 4 \\
		x = 2 \\
		P(x = 2) = \frac{ \binom{4}{2} \binom{6}{1}}{\binom{10}{3}} = \frac{3}{10} = 0.3 \textit{ o 30\%.}
	\end{gather*}




