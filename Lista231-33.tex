\documentclass[12pt,letterpaper]{article}
\usepackage[latin1]{inputenc}
\usepackage[spanish]{babel}
\usepackage{amsmath}
\usepackage{amsfonts}
\usepackage{amssymb}
\usepackage{graphicx}
\usepackage{multicol}
\usepackage{vwcol}
%\setlength{\parindent}{0cm}
\usepackage[dvipsnames]{xcolor}
\usepackage[left=2cm,right=2cm,top=2cm,bottom=2cm]{geometry}
\author{}


\begin{document}

\paragraph{Ejercicio 31} Un promedio de 0.5 clientes por minuto llega a una caja de salida en un almacen. �Cu�l es la probabilidad de que lleguen m�s de 20 clientes a la caja en un intervalo especifico de media hora? \\

\textit{Soluci�n: } \\
X es una variable con distribuci�n exponencial. \\
Datos: \\
- $ P = 0.5 x min $ \\
- $ n = \frac{1}{2}hora = 30 min. $ \\
- $\lambda = np = (0.5)(30) = 15 $\\
- $ x > 20 $ \\
$$ f(x) = \lambda e^{-\lambda x} = 15e^{-15x}$$

$$ \mu = \int_{20}^{\infty} xf(x) = \lim_{b \to \infty} \int_{20}^{b} x15e^{-15x}dx = \lim_{b \to \infty} x(-e^{-15x}) \rvert_{20}^{b} - \int_{20}^{b} -e^{-15x}dx 
$$

$$ \mu = \lim_{b \to \infty} \left( - \frac{x}{e^{15x}} - \frac{1}{15e^{15x}}  \right)  \rvert_{20}^{b}
$$

$$ \mu = \lim_{n \to \infty} \left(  -\frac{b}{e^{15b}} - \frac{1}{15e^{15b}} + \frac{20}{e^{(15)(20)}} + \frac{1}{e^{(15)(20)}}  \right) = 20 + \frac{1}{15} = 20.067
$$

Entonces, 
\begin{center}
$ P(x,t = \frac{1}{2}) = \int_{0}^{\frac{1}{2}} \frac{301}{15}e^{-\frac{301}{15}} = 0.999 $ � $99.9\% $
\end{center}

\paragraph{Ejercicio 32} Determine la media y la varianza para todas las distribuciones vistas en clase. \\
\par

\setlength{\parindent}{1cm}
\textcolor{Violet}{- Soluci�n:} Media Poisson 
\newline
\par 

\setlength{\parindent}{1cm}
Sea x una variable aleatoria con distribuci�n de Poisson
\par 
\begin{center}
$$E[x] = \sum_{j = 0}^\infty jp(1 - p)^{j}$$
$$E[x] = p(1 - p)\sum_{j = 0}^\infty j(1 - p)^{j - 1}$$
$$E[x] = -p(1 - p)\sum_{j = 0}^\infty \dfrac{d}{dp}(1 - p)^{j}$$
\newline
Ya que una serie de potencias puede ser diferenciada t�rmino a t�rmino, se sigue que
\newline
$$E[x] = -p(1 - p)\dfrac{d}{dp}\sum_{j = 0}^\infty (1 - p)^{j}$$
\newline
Utilizando la f�rmula para una progresi�n geom�trica, vemos que
\newline
$$E[x] = -p(1 - p)\dfrac{d}{dp}\left(\dfrac{1}{p}\right) = -p(1 - p)\left(\dfrac{-1}{p^{2}}\right)$$
\newline
\end{center}

\begin{flushleft}
Lo que nos da como resultado
\newline
$$E[x] = \dfrac{1 - p}{p}$$
\newline
\textcolor{Violet}{- Soluci�n:} Media Distribuci�n Binomial \hfill
\newline
\newline
Sea x una variable aleatoria con una distribuci�n binomial \hfill
\newline
\newline
Para $n = 1 \;\;\; x$ asume los valores $0$ y $1$ con probabilidades $1 - p$ y $p$ respectivamente
\newline
$$E[x] = 0p(x = 0) + 1p(x = 1) = p$$
Calculando para cualquier $n \geq 1$ 
\newline
$$E[x] = \sum_{j = 0}^\infty j\binom{n}{j}p^{j}(1 - p)^{n - j}$$
\newline
Para calcular esto observamos que
\newline
$$j\binom{n}{j} = \dfrac{jn!}{j!(n - j)!}$$
\newline
$$j\binom{n}{j} = \dfrac{n(n - 1)!}{(j - 1)![(n - 1)- (j - 1)]!}$$
\newline
$$j\binom{n}{j} = n\binom{n - 1}{j - 1}$$
\newline
Entonces
\newline
$$E[x] = n\sum_{j = 0}^n \binom{n - 1}{j - 1}p^{j}(1 - p)^{n - j}$$
\newline
Si $i = j - 1$ vemos que
\newline
$$E[x] = np\sum_{i = 0}^{n - 1} \binom{n - 1}{i}p^{i}(1 - p)^{n - i - 1}$$
\newline
Por el teorema del binomio
\newline
$$\sum_{i = 0}^{n - 1} \binom{n - 1}{i}p^{i}(1 - p)^{n - i - 1} = [p + (1 - p)^{n - 1}] = 1$$
\newline
Por lo que 
\newline
$$E[x] = np$$
\newline
\newline
\end{flushleft}

\paragraph{Ejercicio 33 } Demuestre que las distribuciones vistas en clase son de probabilidad.
\newline
\par
{\setlength{\parindent}{1cm}
\textcolor{Violet}{Soluci�n:}
\newline
\par 
a) Distribuci�n Binomial
\newline
$$P(\Omega) = 1$$
\newline
$$P(\Omega) = \sum_{x = 0}^{n}f(x) = \sum_{x = 0}^{n} \binom{n}{x}p^{x}q^{n - x} = 1 $$
\par 
Sabemos que 
\newline
$$(a + b)^{n} = \sum_{x = 0}^{n} \binom{n}{x}a^{x}b^{n - x} \qquad \qquad  \forall  \;\; a,b\; \in \; \mathbb{R}$$
\par
si $a = p$ y $b = q$
\newline
$$(p + q)^{n} = \sum_{x = 0}^{n} \binom{n}{x}p^{x}q^{n - x}$$
\newline
$$(1)^{n} = \sum_{x = 0}^{n} \binom{n}{x}p^{x}q^{n - x} \qquad \qquad ya \; que \; \; p + q = 1$$
\newline
$$1= \sum_{x = 0}^{n} \binom{n}{x}p^{x}q^{n - x}$$
\newline
\par 
\noindent\textcolor{Violet}{Soluci�n:}
\newline
\par 
b) Distribuci�n Binomial Negativa
\newline
\par 
\textcolor{Violet}{Soluci�n:}
\newline
$$1 = \sum_{n = x}^{\infty} \binom{n - 1}{x - 1}p^{x}(1 - p)^{n - x} = \sum_{n = x}^{\infty} \dfrac{(n - 1)!}{(x - 1)!(n - x)!}p^{x}(1 - p)^{n - x} = \dfrac{p^{x}}{(x - 1)!}\sum_{n = 0}^{\infty} \dfrac{(n + x - 1)!}{n!}(1 - p)^{n}$$
\newline
\par
Entonces podemos cambiar nuestra hip�tesis de inducci�n (HI) suponiendo que
\newline
\begin{equation}
HI = \sum_{n = 0}^{\infty} \dfrac{(n + x - 1)!}{n!}(1 - p)^{n} = \dfrac{(x - 1)!}{p^{x}}
\end{equation}
\par 
Demostrar que (1) es verdadera para $x+ 1$
\newline
\begin{equation}
\sum_{n = 0}^{\infty} \dfrac{(n + x)!}{n!}(1 - p)^{n} = \dfrac{x!}{p^{x + 1}}
\end{equation}
\newline
\par 
Tenemos que:
\newline
$$\sum_{n = 0}^{\infty} \dfrac{(n + x)!}{n!}(1 - p)^{n}  = \sum_{n = 0}^{\infty} (n + x)\dfrac{(n + x - 1)!}{n!}(1 - p)^{n}$$ 
\begin{equation}
= \sum_{n = 1}^{\infty} n\dfrac{(n + x - 1)!}{n!}(1 - p)^{n} + x\sum_{n = 0}^{\infty} x\dfrac{(n + x - 1)!}{n!}(1 - p)^{n}
\end{equation}
\newline
$$HI = \sum_{n = 1}^{\infty} n\dfrac{(n + x - 1)!}{n!}(1 - p)^{n} + x\dfrac{(x - 1)!}{p^{x}}$$ 
\newline
\par 
Trabajaremos ahora con
\newline
$$\sum_{n = 1}^{\infty} n\dfrac{(n + x - 1)!}{n!}(1 - p)^{n}$$
\newline
\par 
Tenemos que 
\newline
$$\sum_{n = 1}^{\infty} n\dfrac{(n + x - 1)!}{n!}(1 - p)^{n} = (1 - p)\sum_{n = 1}^{\infty} n\dfrac{(n + x - 1)!}{n!}(1 - p)^{n - 1}$$
\newline
\begin{equation}
= (1 - p)\sum_{n = 0}^{\infty} -\dfrac{d}{dp}\left(\dfrac{(n + x - 1)!}{n!}(1 - p)^{n}\right) = -(1 - p)\dfrac{d}{dp}\sum_{n = 0}^{\infty} \left(\dfrac{(n + x - 1)!}{n!}(1 - p)^{n}\right)
\end{equation}
\newline
\par 
{\setlength{\parindent}{5cm}
HI}
\newline
$$ = -(1 - p)\dfrac{d}{dp}\left(\dfrac{(x - 1)!}{p^{x}}\right) = (1 - p) \dfrac{x!}{p^{x + 1}}$$
\newline
\par 
Insertando (3) en (4) obtenemos:
\newline
$$\sum_{n = 0}^{\infty} \dfrac{(n + x)!}{n!}(1 - p)^{n} = (1 - p)\dfrac{x!}{p^{x + 1}} + \dfrac{x!}{p^{x}} = \dfrac{(1 - p)x! + p(x!)}{p^{x + 1}} = \dfrac{x!}{p^{x + 1}}$$
\newline
\par 
Y por lo tanto queda demostrado (2) lo que implica que la proposici�n es verdadera.
\newline
\newline
\par 
\noindent c) Distribuci�n Geom�trica
\newline
\newline
\noindent \textcolor{Violet}{Soluci�n:}
\newline
$$P(\Omega) = 1$$
\newline
$$P(\Omega) = \sum_{x = 1}^{n}f(x) = \sum_{x = 1}^{n} pq^{x - 1}$$
\newline
$$P(\Omega) = pq^{0} + pq^{1} + pq^{2} + pq^{3} + pq^{4} + \cdots + pq^{n - 1}$$
\newline
$$P(\Omega) = p(q^{0} + q^{1} + q^{2} + q^{3} + q^{4} + \cdots + q^{n - 1})$$
\newline
$$S_{n} = r^{0}a_{1} + r^{1}a_{1} + r^{2}a_{1} + r^{3}a_{1} + \cdots + r^{n}a_{1}$$
\newline
$$rS_{n} =  r^{1}a_{1} + r^{2}a_{1} + r^{3}a_{1} + \cdots + r^{n + 1}a_{1}$$
\newline
$$S_{n} - rS_{n} = a_{1} - r^{n + 1}a_{1}$$
\newline
$$S_{n}(1 - r) = a_{1}\dfrac{1 - r^{n + 1}}{1 - r}$$
\newline
$$S_{n \to \infty} =  \lim_{n \to \infty} a_{1}\dfrac{1 - r^{n + 1}}{1 - r}$$
\newline
$ si \qquad r < 1$
\newline
$$= \dfrac{a_{1}}{1 - r}$$
\newline
$$S_{n \to \infty} = \dfrac{1(1 - q^{n + 1})}{1 - q}$$
\newline
$$S_{n \to \infty} = \dfrac{1}{1 - q} = \dfrac{1}{p}$$
\newline
\newline
\par 
\noindent d) Distribuci�n Hipergeom�trica
\newline
\newline
\noindent \textcolor{Violet}{Soluci�n:}
\\\\
\par 
\noindent e) Distribuci�n de Poisson
\newline
\newline
\noindent \textcolor{Violet}{Soluci�n:}
\\

\begin{multicols}{2}
$$P(\Omega) = 1$$
\newline
$$1 = \sum_{x = 0}^{n}\binom{n}{x}p^{x}q^{n - x}$$
\newline
$$1 = \sum_{x = 0}^{n}\binom{n}{x}p^{x}(1 - p)^{n - x}$$
\newline
$$1 = \sum_{x = 0}^{n}\binom{n}{x}\left( \dfrac{\lambda}{n}\right)^{x}\left(1 - \dfrac{\lambda}{n}\right)^{n - x}$$
\newline
$$1 = \sum_{x = 0}^{n}\dfrac{n(n - 1)(n - 2)\ldots(n - x + 1)(n - x)!}{(n - x)!x!}\dfrac{\lambda^{x}}{n^{x}}\left( 1 - \dfrac{\lambda}{n} \right) ^{n} \left( 1 - \dfrac{\lambda}{n} \right) ^{-x}$$
\newline
$$1 = \sum_{x = 0}^{n}\dfrac{\dfrac{n}{n}\left(1 - \dfrac{1}{n}\right)\left(1 - \dfrac{2}{n}\right)\left(1 - \dfrac{3}{n}\right)\ldots\left(1 - \dfrac{x - 1}{n}\right)}{x!}\lambda\left[ 1 - \dfrac{\lambda}{n} \right] ^{\dfrac{-n}{\lambda}} \left( 1 - \dfrac{\lambda}{n} \right) ^{-x}$$
\newline
$$ si\qquad n \to \infty$$
\newline
$$1 = \sum_{x = 0}^{n}\dfrac{1(\lambda)^{x}(1)(e^{-\lambda})}{x!}$$
\newline
$$1 = \sum_{x = 0}^{n}\dfrac{(\lambda)^{x}(e^{-\lambda})}{x!}$$
\newline
\newline
\newline
\newline
\newline
$Nota:$
\begin{flushright}
$y = \left(1 - \dfrac{\lambda}{n}\right)^{\dfrac{-n}{\lambda}}$
\vspace{3mm}
\par 
$ si\qquad n \to \infty$
\vspace{3mm}
\par 
$ln(y) = ln\left(1 - \dfrac{\lambda}{n}\right)^{\dfrac{-n}{\lambda}}$
\vspace{5mm}
\par 
$ln(y) = -\dfrac{n}{\lambda}ln\left(1 - \dfrac{\lambda}{n}\right)$
\vspace{5mm}
\par 
$ln(y) = \dfrac{ln\left(1 - \dfrac{\lambda}{n}\right)}{-\dfrac{\lambda}{n}}$
\vspace{5mm}
\par 
$ln(y) = \dfrac{\dfrac{\dfrac{-\lambda}{n^{2}}}{1 - \dfrac{\lambda}{n}}}{\dfrac{-\lambda}{n^{2}}}  $
\vspace{5mm}
\par 
$ln(y) = \dfrac{\dfrac{-\lambda}{n^{2}}\left( 1 - \dfrac{\lambda}{n}\right)}{\dfrac{-\lambda}{n^{2}}} $
\vspace{5mm}
\par  
$ln(y) = 1 - \dfrac{\lambda}{n} $
\vspace{5mm}
\par 
$ln(y) = 1 $
\vspace{5mm}
\par  
$ln(y) = e^{1} $
\vspace{5mm}
\par  
$ln(y) = e $
\vspace{5mm}
\par  
\end{flushright}

\end{multicols}
\vspace{2cm}

\par 
\noindent f) Distribuci�n Exponencial
\newline
\newline
\noindent \textcolor{Violet}{Soluci�n:}
\\
$$P(\Omega) = 1$$
\newline
$$f(x) = \mu e^{-\mu x} \qquad \qquad x > 0$$
\newline
$$\int_{-\infty}^{\infty}f(x)dx = \int_{-\infty}^{0}f(x)dx + \int_{0}^{\infty}f(x)dx = \int_{0}^{\infty}\mu e^{-\mu x}dx $$
\newline
$$=  \lim_{h \to \infty} \int_{0}^{h}\mu e^{-\mu x}dx =  \lim_{h \to \infty} - e^{-\mu x} \left. \ \textcolor{white}{\dfrac{1}{2}}\right| _{0}^{h} =  \lim_{h \to \infty} - e^{-\mu h} - ( - e^{- \mu (0)}) = e^{0} = 1 $$
}




\end{document}