\section{Lista 2}
\begin{enumerate}
	\item Se sabe que 10\% de los vasos producidos por cierta máquina tienen algún defecto. Si se seleccionan 10 vasos por ésta máquina, ¿cuál es la probabilidad de que ninguno este defectuoso?, ¿cuántos se esperaría encontrar defectuosos?
	\\\textbf{Solución}
	\\\text{La variable tiene una distribución binomial} \\
	\begin{gather*}
		n = 10 \\
		p = 0.1 \\
		q = 0.9 \\
		P(x = 0) = \binom{n}{x}p^xq^{n-x} = \binom{10}{0}(0.1)^0(0.9)^{10-0} = 0.3486784401 \textit{ o 34.86\%.} \\
	\end{gather*}
	\begin{gather*}
		\mu = np = (10)(0.1) = 1
	\end{gather*}
	
	\item Un laberinto para ratas tiene un corredor recto, y al final una bifurcación; en la bifurcación, la debe ir a la derecha o a la izquierda. Suponer que se colocan 10 ratas,  en el laberinto, de una en una. Si cada rata toma al azar una de las dos alternativas del camino. ¿Cuál es la probabilidad de que cuando menos 9 vayan al mismo lado?
	\\\textbf{Solución}
	\\\text{La variable tiene una distribución binomial} \\
	\begin{gather*}
		n = 10 \\
		p = 0.5 \\
		q = 0.5 \\
		P(x = 9) = \binom{10}{9}(0.5)^x(0.5)^{n-x} + \binom{10}{10}(0.5)^x(0.5)^{n-x} = 0.0107421875 \textit{ o 1.07\%.} 
	\end{gather*}
    \item En una “prueba de tortura” se enciende y se apaga un interruptor eléctrico hasta que este falla. Si la probabilidad es 0.001 de que el interruptor falle en cualquier momento en que este encendido o apagado, cual es la probabilidad de que el interruptor no falle durante las primeras 800 veces que se enciende o apague?
    \\\textbf{Solución}
    \\\text{La variable es de Poisson}
    \begin{gather*}
    n=800\\
    p=0.001\\
    q=0.999\\
    \lambda = np = 800(0.001) = 0.8\\
    P(x=0) = \frac{{\lambda}^{x}{e}^{-\lambda}}{x!}= \frac{(0.8)^{0}{e}^{-\lambda}}{0!} = \frac{1{e}^{-.8}}{1} = 0.4493289641 \text{ ó } 44.93\%
    \end{gather*}
    \item Un ingeniero de control de calidad inspecciona una muestra tomada al azar de dos calculadoras portátiles de cada lote de 18 unidades que llega y acepta el lote si ambas están en buenas condiciones de funcionamiento; en caso contrario, se inspecciona todo el lote y el costo se carga al distribuidos. ¿Cuál es la probabilidad de que este lote sea aceptado sin mayor inspección si contiene...
    \begin{enumerate}
        \item Cuatro calculadoras en mal estado?
        \\\textbf{Solución}
        \\\text{La variable es de hipergeometrica}
        \begin{gather*}
        N=18\\
        n=2\\
        k=18-4=14\\
        x=2\\
        P(2) =\frac{\binom{k}{x}\binom{N-k}{n-x}}{\binom{N}{n}} = \frac{\binom{14}{2}\binom{4}{0}}{\binom{18}{2}} = \frac{91}{153} = 0.5947712418 \text{ ó } 59.47\%
        \end{gather*}
        \item Ocho calculadoras en malas condiciones de funcionamiento?
        \\\textbf{Solución}
        \\\text{La variable es de hipergeometrica}
        \begin{gather*}
        N=18\\
        n=2\\
        k=18-8=10\\
        x=2\\
        P(2) =\frac{\binom{k}{x}\binom{N-k}{n-x}}{\binom{N}{n}}= \frac{\binom{10}{2}\binom{8}{0}}{\binom{18}{2}} = \frac{45}{153} =\frac{5}{17} =.294117647 \text{ ó } 29.41\%
        \end{gather*}
    \end{enumerate}
    \item Un examen de opción múltiple consta de ocho preguntas y tres respuestas a cada pregunta. Si un estudiante responde a cada pregunta tirando un dado y marca la primera respuesta si obtiene un 1 o un 2, la segunda respuesta si obtiene un 3 o un 4, y la tercera respuesta si obtiene un 5 o un 6, ¿Cuál es la probabilidad de que logre exactamente cuatro respuestas correctas?
    \\\textbf{Solución}
    \\\text{La variable es de binomial}
    \begin{gather*}
    n=8\\
    p=\frac{1}{3}\\
    q=\frac{2}{3}\\
    P(4) =\binom{n}{x}\left(p\right)^{x}\left(q\right)^{n-x}= \binom{8}{4}\left(\frac{1}{3}\right)^{4}\left(\frac{2}{3}\right)^{4}=0.1707056851 \text{ ó } 17.07\%
    \end{gather*}
    \item Si el $40\%$ de los alumnos se volvieran agresivos en un periodo de 2 horas después de haber ingerido algún liquido en el Sportaco, determine la probabilidad de que exactamente seis de los 15 alumnos que han ingerido algún líquido se vuelvan agresivos en el periodo de 2 horas.
    \\\textbf{Solución}
    \\\text{La variable es de binomial}
    \begin{gather*}
    n=15\\
    p=0.4\\
    q=0.6\\
    P(6) = \binom{n}{x}\left(p\right)^{x}\left(q\right)^{n-x}=\binom{15}{6}(0.4)^{6}(0.6)^{9}=.2065976053 \text{ ó } 20.65\%
    \end{gather*}
    \item Un jurado de 7 jueces debe decidir entre 2 finalistas quien es la ganadora de un concurso de belleza, para lo cual bastara una mayoría de los jueces. Suponga que 4 jueces voten por María y que los otros 3 voten por Susana. Si se seleccionan al azar 3 jueces y se les pregunta por quien van a votar, ¿cuál es la probabilidad de que la mayoría de los jueces de la muestra estén a favor de María?
    \\\textbf{Solución}
    \\\text{La variable es de hipergeometrica}
    \begin{gather*}
    N=7\\
    n=3\\
    k=4\\
    x=2, 3\\
    p(2, 3) = \sum _{x=2}^{3} \frac{\binom{k}{x}\binom{N-k}{n-x}}{\binom{N}{n}} = \frac{\binom{4}{2}\binom{3}{1}}{\binom{7}{3}} + \frac{\binom{4}{3}\binom{3}{0}}{\binom{7}{3}} = \frac{22}{35} =0.6285714286 \text{ ó } 62.85\%
    \end{gather*}
    \item Se ha observado que el transito promedio de automóviles en determinado punto de un camino rural es de 3 por hora. Suponga que los instantes en que pasan los mismos son independientes, haciendo que x represente el numero de los que pasan por este punto en un intervalo de 20 minutos, calcule la probabilidad de $P(x > 2)$
    \\\textbf{Solución}
    \\\text{La variable es de Poisson}
    \begin{gather*}
    \lambda = np = 1 \tag{ cada 20 min}\\
    x>2\\
    p(x>2) =\sum _{x=3}^{\infty }\frac{{\lambda}^{x}{e}^{-\lambda}}{x!}=\sum _{x=3}^{\infty }\frac{{1}^{x}{e}^{-1}}{x!}=1 - \sum _{x=0}^{2}\frac{{1}^{x}{e}^{-1}}{x!} = 1- \left( \frac{{e}^{-1}}{0!}+\frac{{e}^{-1}}{1!}+\frac{{e}^{-1}}{2!} \right) \\
    = 0.08030139707 \text{ ó } 8.03\%
    \end{gather*}
    \item En determinada planta manufacturera han ocurrido accidentes a razón de 1 cada 2 meses. suponiendo que ocurren en forma independiente, Cual es el numero esperado de accidentes al año?
    \\\textbf{Solución}
    \\\text{La variable es de Poisson}
    \begin{gather*}
    \lambda = 1 \tag{ cada 2 meses}\\
    \lambda = \frac{(1 \text{ accidentes})(12 \text{ meses})}{2 \text{ meses}} = 6 \text{ accidentes}\\
    \end{gather*}
    \item En Chilpancingo, la incompatibilidad se da como la razón o motivo legal en el 70\% de todos los casos de divorcio. Obtenga la probabilidad de que 5 de los 6 siguientes divorcios archivados de esta ciudad argumentan incompatibilidad como motivo principal.
    \\\textbf{Solución}
    \\La variable es binomial
    \begin{gather*}    
    n = 6\\
    p = 0.70 \\
    q = 0.30 \\
    P(5) = \binom{6}{5}(0.7)^5(0.3)^1 = 0.302526 \text{ ó }  30.35\%
    \end{gather*}
    \item Un psicólogo asevera que sólo el 50\%de todos los alumnos del último semestre de vocacional, capaces de desempeñar trabajos a nivel superior, asisten en realidad al nivel superior. Suponiendo verdadera esta afirmación, obtenga las probabilidades de que, entre 18 alumnos capaces de desempeñar trabajos a nivel superior, exactamente 10 asistan a ESCOM.
    \\\textbf{Solución}
    \\ No se puede resolver, pues no se puede saber cuántos de los que están en superior, están en ESCOM.
    \item Suponga que el 40\% de los empleados a destajo de la empresa ACME están a favor de tener representación sindical y que se entrevista a una muestra aleatoria de 10 de ellos y se les solicita una respuesta anónima.\\¿Cuál es la probabilidad de que la mayoria de los que respondan estarán a favor de la representación sindical?
    \\\textbf{Solución}
    \\\text{La variable es de poisson}
    \begin{gather*}
    \lambda=np=(10)(0.4) = 4\\
    x>6\\
    P(x) =1-\sum _{x=0}^{5}\frac{{\lambda}^{x}{e}^{-\lambda}}{x!}=1 - \sum _{x=0}^{5}\frac{{4}^{x}{e}^{-4}}{x!} =  0.214869613 \text{ ó } 21.48\%
    \end{gather*}
    \item Un profesor de ESCOM selecciona al azar a 3 alumnos de un grupo de 10 para aprobarlos. Suponiendo que el semestre anterior aprobó cuatro de esos 10 alumnos, determine la probabilidad de que exactamente 2 de los 3 alumnos hayan aprobado en el semestre anterior.
    \\\textbf{Solución}
    \\\text{La variable tiene una distribución hipergeométrica} \\
    \begin{gather*}
    N = 10 \\
    n = 3 \\
    k = 4 \\
    x = 2 \\
    P(x = 2) = \frac{ \binom{4}{2} \binom{6}{1}}{\binom{10}{3}} = \frac{3}{10} = 0.3 \textit{ o 30\%.}
    \end{gather*}
    \item En promedio, de cada 500 cervezas servidas en el Sportaco dos salen defectuosas, cuál es la probabilidad de que en un lote especifico de 100 cervezas no haya ninguna defectuosa?
    \\\textbf{Solución}
    \begin{gather*}
    n = 100 \text{ cervezas}\\
    p = \frac{1}{250} \text{ (Regla de tres)}\\
    \lambda = np = \frac{2}{5}\\
    P(x=0) = \frac{{\lambda}^{x}{e}^{-\lambda}}{x!}=\frac{(0.4)^{0}(e)^{-0.4}}{0!}= 0.670320046 \text{ ó } 67.03\%\\
    \end{gather*}
    \setcounter{enumi}{23}
    \item Supóngase que la cantidad real de café instantáneo colocada por una máquina llenadora en frascos de "6 onzas" es ua variable aleatoria que tiene distribución normal con varianza $0.0025$ de onza. Si sólo el $3\%$ de los frascos va a contener menos de 6 onzas de café, ¿cuál debe ser el contenido medio de estos frascos?\\
    \textbf{Solución}\\
    \textbf{La variable es de distribución normal.}
    \begin{enumerate}
        \item Si la probabilidad de que el frasco contenga menos de 6 onzas de café es de $3\%$, entonces:
        \begin{gather*}
        \int_{-\infty}^{x}f(x)dx=0.03=F(x)-F(-\infty)=F(x)-0
        \end{gather*}
        \item Obervando las tablas para el área bajo la curva de la función de distribución normal estándar, se obtiene que x = -0.52, pero$x = Z_{max}$, entonces\\
        \begin{gather*}
        Z_{max} = \frac{b-\mu}{\sigma}\\
        -0.52=\frac{6 -\mu}{0.05}\\
        \mu = 6.026
        \end{gather*}
    \end{enumerate}
    
    \item Si el $23\%$ de todos los pacientes con presión sanguínea elevada tienen efectos colaterales nocivos por la ingestión de cierto medicamento, utilie la aproximación normal para obtener la probabilidad de que entre 120 de estos pacientes tratados con este medicamento unas 32 presentarán efectos colaterales nocivos
    
    \textbf{Solución}\\
    \textbf{La variable es de distribución normal.}
    \begin{enumerate}
        \item Determinar la media y desviación estándar:
        \begin{gather*}
        \mu =np=(120)(0.23)=27.6\\
        \sigma=\sqrt{\sigma^{2}}	= \sqrt{npq}=\sqrt{(120)(0.23)(0.77)} = \pm 4.609989
        \end{gather*}
        \item Calcular la probabilidad por aproximación normal. Calculando $P(31.5\leq x\leq32.5)$ en lugar de $P(32\leq x\leq32)$\\
        \begin{gather*}
        P(x=32)= P(31.5\leq x\leq32.5)
        \end{gather*}
        \item Calcular $Z_{max}$ y $Z_{min}$:
        \begin{gather*}
        Z_{max}=\frac{32.5-27.6}{4.609}=1.063137\\
        Z_{min}=\frac{31.5-27.6}{4.609}=0.846117\\\\
        \Rightarrow  P(31.5\leq x\leq32.5) = F(1.063137)-F(0.846117)\\
        = 0.8554276993-0.7995458057=0.0558818936\\
        P(31.5\leq x\leq32.5) = 0.0558818936 \text{ ó } 0.558\%
        \end{gather*}
    \end{enumerate}
    
    
    \item Se ha ajustado el proceso de fabricación de un tornillo de precisión de manera que la longitud promedio de los tornillos sea de 13cm. Por supuesto no todos los tornillos tiene una longitud exacta de 13cm, debido a fuentes aleatorias de variabilidad. La desviación estaándar de la longitud de los tornillos de 0.1cm y se sabe que la distribución de las longitudes tiene una forma normal. Determine la probabilidad de que un tornillo elegindo al azar tenga una longitud de entre 13.0 y 13.2cm.
    
    \textbf{Solución}\\
    \textbf{La variable es de distribución normal.}
    \begin{enumerate}
        
        \item Calcular la probabilidad $P(13\leq x\leq 13.2)$ \\
        \begin{gather*}
        \mu = 13\\
        \sigma = 0.1\\
        Z_{max} = \frac{13.2-13}{0.1} = 2\\
        Z_{min} = \frac{13-13}{0.1} = 0\\
        \Rightarrow P(13\leq x\leq 13.2)=P0\leq x\leq 2) = F(2) - F(0)\\
        0.9772498670-0.4999999990 = 0.477249868\\\\
        P(13\leq x\leq 13.2) =  0.4772 \text{ ó } 47.72\%
        \end{gather*}
    \end{enumerate}
    
    \item El número promedio de solicitudes de servicio que se reciben en un departamento de reparación de maqunaria por cada turno de 8 horas es de 10. Deterine la probabilidad de que se reciban más de 15 solicitides en un turno de 8 horas elegido al azar.
    
    \textbf{Solución}\\
    \textbf{La distribución es de Poisson.}
    \begin{enumerate}
        
        \item Calcular la probabilidad $P( x >15)$ \\
        \begin{gather*}
        \lambda = 10\\
        P( x >15) = \sum_{x=15}^{\infty}\frac{\lambda^{x}e^{-\lambda}}{x!}\\
        =1- \sum_{x=0}^{15}\frac{\lambda^{x}e^{-\lambda}}{x!}\\
        =1- \sum_{x=0}^{15}\frac{(10)^{x}e^{-10}}{x!}\\
        =0.0487404033 \text{ ó } 4.87\%
        \end{gather*}
    \end{enumerate}
    
    \item Un embarque de 10 máquinas incluye una defectuosa. Si se eligen 7 máquinas al azar de ese embarque, ¿cuál es la probabilidad de que ningua de las 7 esté defectuosa?
    
    \textbf{Solución}\\
    \textbf{La distribución es hipergeométrica}
    \begin{enumerate}
        
        \item Calcular la probabilidad $P(0)$ \\
        \begin{gather*}
        N = 10\\
        n = 7\\
        K=9\\
        x=7\\\\
        P(0)=\frac{\binom{9}{7}\binom{1}{0}}{\binom{10}{7}} = \frac{3}{10} \text{ ó }30\%
        \end{gather*}
    \end{enumerate}
    
    \item Suponga que la proporción de máquinas defectuosas en una operación de ensamble es de 0.01 y que se incluye una muestra de 200 de ellas en un embarque específico, ¿cuál es la probabilidad de que cuando mucho 3 máquinas estén defectuosas?
    \textbf{Solución}\\
    \textbf{La distribución es de Poisson}
    \begin{enumerate}
        \item Calcular la probabilidad $P(x<4)$ \\
        \begin{gather*}
        \lambda = np= 200(0.01) = 2\\
        Px<4()=\sum_{x=0}^{3}\frac{\lambda^{x}e^{-\lambda}}{x!}\\
        = \sum_{x=0}^{3}\frac{(2)^{x}e^{-2}}{x!}=0.8571234605\text{ ó } 85.71\%\\
        \end{gather*}
    \end{enumerate}
    
    \item Un promedio de 0.5 clientes por minuto llega a una caja de salida en un almacén, ¿cuál es la probabilidad d eque lleguen 5 o más clientes en un intervalo dado de 5 minutos.
    \textbf{Solución}\\
    \textbf{La distribución es de Poisson}
    \begin{enumerate}
        \item Calcular la probabilidad $P(x<4)$ \\
        \begin{gather*}
        \lambda = np=(0.5)(5) = 2.5\\
        Px<4()=\sum_{x=5}^{\infty}\frac{\lambda^{x}e^{-\lambda}}{x!}\\
        = \sum_{x=0}^{3}\frac{(2.5)^{x}e^{-2.5}}{x!}=0.1088219811\text{ ó } 10.88\%\\
        \end{gather*}
    \end{enumerate}
    \item Un promedio de 0.5 clientes por minuto llega a una caja de salida en un almacen. ¿Cuál es la probabilidad de que lleguen más de 20 clientes a la caja en un intervalo especifico de media hora? \\
    
    \textit{Solución: } \\
    X es una variable con distribución exponencial. \\
    Datos: \\
    - $ P = 0.5 x min $ \\
    - $ n = \frac{1}{2}hora = 30 min. $ \\
    - $\lambda = np = (0.5)(30) = 15 $\\
    - $ x > 20 $ \\
    $$ f(x) = \lambda e^{-\lambda x} = 15e^{-15x}$$
    
    $$ \mu = \int_{20}^{\infty} xf(x) = \lim_{b \to \infty} \int_{20}^{b} x15e^{-15x}dx = \lim_{b \to \infty} x(-e^{-15x}) \rvert_{20}^{b} - \int_{20}^{b} -e^{-15x}dx 
    $$
    
    $$ \mu = \lim_{b \to \infty} \left( - \frac{x}{e^{15x}} - \frac{1}{15e^{15x}}  \right)  \rvert_{20}^{b}
    $$
    
    $$ \mu = \lim_{n \to \infty} \left(  -\frac{b}{e^{15b}} - \frac{1}{15e^{15b}} + \frac{20}{e^{(15)(20)}} + \frac{1}{e^{(15)(20)}}  \right) = 20 + \frac{1}{15} = 20.067
    $$
    
    Entonces, 
    \begin{center}
        $ P(x,t = \frac{1}{2}) = \int_{0}^{\frac{1}{2}} \frac{301}{15}e^{-\frac{301}{15}} = 0.999 $ ó $99.9\% $
    \end{center}
    \item Determine la media y la varianza para todas las distribuciones vistas en clase.
    Media Poisson 
    Sea x una variable aleatoria con distribución de Poisson
    \begin{center}
        $$E[x] = \sum_{j = 0}^\infty jp(1 - p)^{j}$$
        $$E[x] = p(1 - p)\sum_{j = 0}^\infty j(1 - p)^{j - 1}$$
        $$E[x] = -p(1 - p)\sum_{j = 0}^\infty \dfrac{d}{dp}(1 - p)^{j}$$
        Ya que una serie de potencias puede ser diferenciada término a término, se sigue que
        $$E[x] = -p(1 - p)\dfrac{d}{dp}\sum_{j = 0}^\infty (1 - p)^{j}$$
        Utilizando la fórmula para una progresión geométrica, vemos que
        $$E[x] = -p(1 - p)\dfrac{d}{dp}\left(\dfrac{1}{p}\right) = -p(1 - p)\left(\dfrac{-1}{p^{2}}\right)$$
    \end{center}
        Lo que nos da como resultado
        $$E[x] = \dfrac{1 - p}{p}$$
        Media Distribución Binomial
        Sea x una variable aleatoria con una distribución binomial
        Para $n = 1 \;\;\; x$ asume los valores $0$ y $1$ con probabilidades $1 - p$ y $p$ respectivamente
        $$E[x] = 0p(x = 0) + 1p(x = 1) = p$$
        Calculando para cualquier $n \geq 1$ 
        $$E[x] = \sum_{j = 0}^\infty j\binom{n}{j}p^{j}(1 - p)^{n - j}$$
        Para calcular esto observamos que
        $$j\binom{n}{j} = \dfrac{jn!}{j!(n - j)!}$$
        $$j\binom{n}{j} = \dfrac{n(n - 1)!}{(j - 1)![(n - 1)- (j - 1)]!}$$
        $$j\binom{n}{j} = n\binom{n - 1}{j - 1}$$
        Entonces
        $$E[x] = n\sum_{j = 0}^n \binom{n - 1}{j - 1}p^{j}(1 - p)^{n - j}$$
        Si $i = j - 1$ vemos que
        $$E[x] = np\sum_{i = 0}^{n - 1} \binom{n - 1}{i}p^{i}(1 - p)^{n - i - 1}$$
        Por el teorema del binomio
        $$\sum_{i = 0}^{n - 1} \binom{n - 1}{i}p^{i}(1 - p)^{n - i - 1} = [p + (1 - p)^{n - 1}] = 1$$
        Por lo que 
        $$E[x] = np$$
\end{enumerate}