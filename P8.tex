%La octava parte: problemas 50-56, que son los problemas 24-30 de la lista 2

%Para los problemas de 8-14 de la lista 1:
\begin{enumerate}
	\setcounter{enumi}{23} % Para que empiece en 24
	\item Supóngase que la cantidad real de café instantáneo colocada por una máquina llenadora en frascos de "6 onzas" es ua variable aleatoria que tiene distribución normal con varianza $0.0025$ de onza. Si sólo el $3\%$ de los frascos va a contener menos de 6 onzas de café, ¿cuál debe ser el contenido medio de estos frascos?\\
	\textbf{Solución}\\
	\textbf{La variable es de distribución normal.}
    \begin{enumerate}
        \item Si la probabilidad de que el frasco contenga menos de 6 onzas de café es de $3\%$, entonces:
        \begin{gather*}
            \int_{-\infty}^{x}f(x)dx=0.03=F(x)-F(-\infty)=F(x)-0
    	\end{gather*}
    	\item Obervando las tablas para el área bajo la curva de la función de distribución normal estándar, se obtiene que x = -0.52, pero$x = Z_{max}$, entonces\\
    	\begin{gather*}
    		Z_{max} = \frac{b-\mu}{\sigma}\\
    		-0.52=\frac{6 -\mu}{0.05}\\
		\therefore	\mu = 6.026
    	\end{gather*}
    \end{enumerate}



	\item Si el $23\%$ de todos los pacientes con presión sanguínea elevada tienen efectos colaterales nocivos por la ingestión de cierto medicamento, utilie la aproximación normal para obtener la probabilidad de que entre 120 de estos pacientes tratados con este medicamento unas 32 presentarán efectos colaterales nocivos

	\textbf{Solución}\\
	\textbf{La variable es de distribución normal.}
    \begin{enumerate}
        \item Determinar la media y desviación estándar:
        \begin{gather*}
			\mu =np=(120)(0.23)=27.6\\
			\sigma=\sqrt{\sigma^{2}}	= \sqrt{npq}=\sqrt{(120)(0.23)(0.77)} = \pm 4.609989
		\end{gather*}
    	\item Calcular la probabilidad por aproximación normal. Calculando $P(31.5\leq x\leq32.5)$ en lugar de $P(32\leq x\leq32)$\\
    	\begin{gather*}
    		P(x=32)= P(31.5\leq x\leq32.5)
    	\end{gather*}
    	\item Calcular $Z_{max}$ y $Z_{min}$:
    	\begin{gather*}
    		Z_{max}=\frac{32.5-27.6}{4.609}=1.063137\\
    		Z_{min}=\frac{31.5-27.6}{4.609}=0.846117\\\\
    		\Rightarrow  P(31.5\leq x\leq32.5) = F(1.063137)-F(0.846117)\\
    		= 0.8554276993-0.7995458057=0.0558818936\\
    		\therefore P(31.5\leq x\leq32.5) = 0.0558818936 \text{ ó } 0.558\%
    	\end{gather*}
    \end{enumerate}


	\item Se ha ajustado el proceso de fabricación de un tornillo de precisión de manera que la longitud promedio de los tornillos sea de 13cm. Por supuesto no todos los tornillos tiene una longitud exacta de 13cm, debido a fuentes aleatorias de variabilidad. La desviación estaándar de la longitud de los tornillos de 0.1cm y se sabe que la distribución de las longitudes tiene una forma normal. Determine la probabilidad de que un tornillo elegindo al azar tenga una longitud de entre 13.0 y 13.2cm.

	\textbf{Solución}\\
	\textbf{La variable es de distribución normal.}
    \begin{enumerate}

    	\item Calcular la probabilidad $P(13\leq x\leq 13.2)$ \\
    	\begin{gather*}
		\mu = 13\\
		\sigma = 0.1\\
		Z_{max} = \frac{13.2-13}{0.1} = 2\\
		Z_{min} = \frac{13-13}{0.1} = 0\\
		\Rightarrow P(13\leq x\leq 13.2)=P0\leq x\leq 2) = F(2) - F(0)\\
		0.9772498670-0.4999999990 = 0.477249868\\\\
		\therefore  P(13\leq x\leq 13.2) =  0.4772 \text{ ó } 47.72\%
    	\end{gather*}
    \end{enumerate}

	\item El número promedio de solicitudes de servicio que se reciben en un departamento de reparación de maqunaria por cada turno de 8 horas es de 10. Deterine la probabilidad de que se reciban más de 15 solicitides en un turno de 8 horas elegido al azar.

	\textbf{Solución}\\
	\textbf{La distribución es de Poisson.}
    \begin{enumerate}

    	\item Calcular la probabilidad $P( x >15)$ \\
    	\begin{gather*}
		\lambda = 10\\
		P( x >15) = \sum_{x=15}^{\infty}\frac{\lambda^{x}e^{-\lambda}}{x!}\\
		=1- \sum_{x=0}^{15}\frac{\lambda^{x}e^{-\lambda}}{x!}\\
		=1- \sum_{x=0}^{15}\frac{(10)^{x}e^{-10}}{x!}\\
		=0.0487404033 \text{ ó } 4.87\%
    	\end{gather*}
    \end{enumerate}

    \item Un embarque de 10 máquinas incluye una defectuosa. Si se eligen 7 máquinas al azar de ese embarque, ¿cuál es la probabilidad de que ningua de las 7 esté defectuosa?

	\textbf{Solución}\\
	\textbf{La distribución es hipergeométrica}
    \begin{enumerate}

    	\item Calcular la probabilidad $P(0)$ \\
    	\begin{gather*}
		N = 10\\
		n = 7\\
		K=9\\
		x=7\\\\
		P(0)=\frac{\binom{9}{7}\binom{1}{0}}{\binom{10}{7}} = \frac{3}{10} \text{ ó }30\%
    	\end{gather*}
    \end{enumerate}

    \item Suponga que la proporción de máquinas defectuosas en una operación de ensamble es de 0.01 y que se incluye una muestra de 200 de ellas en un embarque específico, ¿cuál es la probabilidad de que cuando mucho 3 máquinas estén defectuosas?
	\textbf{Solución}\\
	\textbf{La distribución es de Poisson}
    \begin{enumerate}
    	\item Calcular la probabilidad $P(x<4)$ \\
    	\begin{gather*}
		\lambda = np= 200(0.01) = 2\\
		Px<4()=\sum_{x=0}^{3}\frac{\lambda^{x}e^{-\lambda}}{x!}\\
		= \sum_{x=0}^{3}\frac{(2)^{x}e^{-2}}{x!}=0.8571234605\text{ ó } 85.71\%\\
    	\end{gather*}
    \end{enumerate}

      \item Un promedio de 0.5 clientes por minuto llega a una caja de salida en un almacén, ¿cuál es la probabilidad d eque lleguen 5 o más clientes en un intervalo dado de 5 minutos.
	\textbf{Solución}\\
	\textbf{La distribución es de Poisson}
    \begin{enumerate}
    	\item Calcular la probabilidad $P(x<4)$ \\
    	\begin{gather*}
		\lambda = np=(0.5)(5) = 2.5\\
		Px<4()=\sum_{x=5}^{\infty}\frac{\lambda^{x}e^{-\lambda}}{x!}\\
		= \sum_{x=0}^{3}\frac{(2.5)^{x}e^{-2.5}}{x!}=0.1088219811\text{ ó } 10.88\%\\
    	\end{gather*}
    \end{enumerate}

\end{enumerate}
