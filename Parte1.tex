\documentclass{article}
\usepackage[utf8]{inputenc}
\usepackage{mathtools}

\title{Parte1}
\author{DBL }
\date{Dicimebre 2016}

\begin{document}

\maketitle

\begin{enumerate}
    \begin{gather*}
         ii)
    \end{gather*}
    \\\textbf{Soluci�n}\\Se verifica para n=1
    \begin{gather*}
        \ (x+y)^1 =   \binom{1}{0}(x^1)(y^0) + \binom{1}{1}(x^0)(y^1)\\
        \ (x+y)^1 =   x +\ y
    \end{gather*}\\
        
    Supongamos 
    \begin{gather*}  
        (x+y)^n =   \sum_{r=0}^{n} \binom{n}{r} x^{n-r}y^{r}
    \end{gather*}
    Y probemos
    \begin{gather*}
        (x+y)^{n+1} = \sum_{r=0}^{n+1} \binom{n+1}{r} x^{n+1-r}y^{r}\\\
    \end{gather*}    
    \begin{gather*}
        (x+y)^{n+1} = (x+y)(x+y)^n = (x+y) \sum_{r=0}^{n+1} \binom{n}{r} x^{n-r}y^{r}=\\\
    \end{gather*}
    \begin{gather*}
        \ x \sum_{r=0}^{n} \binom{n}{r} x^{n-r}y^{r} +  y \sum_{r=0}^{n} \binom{n}{r} x^{n-r}y^{r}=\\\
    \end{gather*}
    \begin{gather*}
         \  \sum_{r=0}^{n} \binom{n}{r} x^{(n-1)-r}y^{r} +  y \sum_{r=1}^{n+1} \binom{n}{r-1} x^{n-(r-1)}y^{r}=\\\
    \end{gather*}
    \begin{gather*}
         \  \binom{n}{0} x^{n+1}y^{0} + \sum_{r=1}^{n} \binom{n}{r} x^{(n+1)-r}y^{r} + \sum_{r=1}^{n} \binom{n}{r-1} x^{(n+1)-r}y^{r} + \binom{n}{n+r-1} x^{n-(r-1)}y^{n+1}=\\\
    \end{gather*}
    \begin{gather*}
         \  x^{n+1} + \sum_{r=1}^{n} \binom{n}{r} x^{(n+1)-r}y^{r} +   \sum_{r=1}^{n} \binom{n}{r-1} x^{n+1-r)}y^{r} + y^{n+1}=\\\
    \end{gather*}
    \begin{gather*}
         \  x^{n+1} + \sum_{r=1}^{n} (\binom{n}{r} + \binom{n}{r-1}) x^{(n+1)-r}y^{r} +  y^{n+1}  =\\\
    \end{gather*}
    \begin{gather*}
         \  x^{n+1} + \sum_{r=1}^{n} \binom{n+1}{r} (x^{(n+1)-r} y^r) =\\\
    \end{gather*}
    \begin{gather*}
         \   \sum_{r=0}^{n+1} \binom{n+1}{r} (x^{(n+1)-r} y^r) \\\
    \end{gather*}
    Por lo tanto se cumple.\\
    
    
     \begin{gather*}
         iii) \sum_{r=0}^{n} r \binom{n}{r} = n2^{n-1}\\ 
    \end{gather*}
    \\\textbf{Soluci�n}\\Partimos de:
    \begin{gather*}
        \   {(1+x)}^n = \sum_{r=0}^{n} \binom{n}{r} (1^{n-r} x^r) =  \sum_{r=0}^{n} \binom{n}{r} x^r \\\
    \end{gather*} Derivando: 
     \begin{gather*}
        \   n{(1+x)}^{n-1} = \sum_{r=1}^{n} r \binom{n}{r} {x^{r-1}} \\\
     \end{gather*} Y con x=1: 
     \begin{gather*}
        \   n2^{n-1} = \sum_{r=1}^{n} r \binom{n}{r} + 0 \\\
    \end{gather*}
    \begin{gather*}
        \   \sum_{r=1}^{n} r \binom{n}{r} =  n2^{n-1} \\\
    \end{gather*}
    
    
    \begin{gather*}
         iv) \sum_{r=0}^{n} \binom{n}{r} {(a-1)}^r = a^n\\ 
    \end{gather*}
    \\\textbf{Soluci�n}\\
    \begin{gather*}
        \  a^n ={(a+1-1)}^n = {[1+(a-1)]}^n = \sum_{r=0}^{n} \binom{n}{r} (1^{n-r}) {(a-1)}^r\\\
    \end{gather*}
    \begin{gather*}
        \  \sum_{r=0}^{n} \binom{n}{r} {(a-1)}^r = a^n\\\
    \end{gather*}
    
    \begin{gather*}
         v) \sum_{r=0}^{n} {\binom{n}{r}}^2 =\binom{2n}{n}\\ 
    \end{gather*}
    \\\textbf{Soluci�n: Sea }\\
    \begin{gather*}
        \sum_{r=0}^{n} {\binom{n}{r}}^2 =\binom{2n}{n}\\ 
    \end{gather*}
     \begin{gather*}
        {(1+x)}^{2n} = {(1+x)}^n * {(1+x)}^n = \sum_{r=0}^{n} (\sum_{r=0}^{n} \binom{n}{j} \binom{n}{r-j} x^k)\\ 
    \end{gather*}
    \begin{gather*}
        {(1+x)}^{2n} = \sum_{k=0}^{2n} \binom{2n}{k} x^k\\ 
    \end{gather*}
    \begin{gather*}
        \binom{2n}{n} = \sum_{j=0}^{n} \binom{n}{j} \binom{n}{n-j} = \sum_{j=0}^{n} {\binom{n}{j}}^2\\ 
    \end{gather*}
    
    \end{enumerate}

	
    \begin{enumerate}
	\setcounter{enumi}{1} 
    \item �De cuantas maneras pueden formarse 5 personas para abordar un autob�s, si dos de las personas se niegan a hacerlo una tras la otra?
    \\\textbf{Soluci�n}\\Todas las combinaciones:
    \begin{gather*}    	
    	Primer lugar  = 5\\
    	Segundo lugar = 4\\
    	Tercer lugar  = 3\\
    	Cuarto lugar  = 2\\
    	Quinto lugar  = 1\\
    	Total         = 120\\
    \end{gather*}\\Combinaciones A-B juntos:\\
    \begin{gather*}    	
    	Primer posicion  = 4\\
    	Segundo posicion = 3\\
    	Tercer posicion  = 2\\
    	Cuarto posicion  = 1\\
    	Formas           = 24\\
    \end{gather*}\\Combinaciones B-A juntos:\\
    \begin{gather*}    	
    	Primer posicion  = 4\\
    	Segundo posicion = 3\\
    	Tercer posicion  = 2\\
    	Cuarto posicion  = 1\\
    	Formas           = 24\\
    	Total formas - "A B juntos" = 120 -24 - 24 = 72\\        
    \end{gather*}\\
    
	\item Dos focos se mantienen encendidos hasta que se funden. Suponer que ninguno dura m�s de 1600 		horas. Definir un espacio muestral adecuado para este experimento, donde se describen los siguientes eventos:\\
	
	 i)Ambos focos duran menos de 1000 horas     \\
	 \textbf{Soluci�n}\\
    \begin{gather*}
        A = \lbrace x,y | 0\leq x,y <1000\rbrace \\
    \end{gather*}\\
    ii)Ning�n foco se funde antes de 1000 horas 
    \\\textbf{Soluci�n}\\
    \begin{gather*}
        B = \lbrace x,y | 1000 < x,y \leq 1600\rbrace \\
    \end{gather*}\\
    iii) El menor tiempo de duraci�n (de los dos) es de 1000 horas
    \\\textbf{Soluci�n}\\
    \begin{gather*}
        C = \lbrace x,y | 1000 \leq x+y \leq 1600\rbrace \\
    \end{gather*}
    
	\item Se inscriben Alejandro, Pedrito y Carlos en una carrera. �Cu�l es la probabilidad de que Alejandro termine antes que Carlos, si todos tienen la misma habilidad y no hay empates? 
    \\\textbf{Soluci�n}
    
    \begin{gather*}
        P(\Omega) = 3! = 6\\
        |\Omega| = {(A,B,C), (B,A,C), (A,C,B), (B,C,A), (C,A,B), (C,B,A)}\\
        |A>C| = {(A,B,C), (A,C,B), (B,A,C)}\\
        P(A>C) = \frac{3}{6} =  \frac{1}{2} = 50\% \\        
    \end{gather*}
    
	\item En un a�o determinado para elecciones nacionales deben elegirse gobernadores para 20 estados. Si se supone que en cada estado hay 2 candidatos (PT y Alianza Zapatista), �cu�l es la probabilidad de que el mismo partido gane en todos los estados?
    \\\textbf{Soluci�n}\\
    \begin{gather*}
    	P(PT \cup AZ)= P(PT) + P(AZ)\\
    	P(PT \cup AZ)= \frac{1}{2^{20}} + \frac{1}{2^{20}} = \frac{2}{2^{20}} = .001907\%\\
    \end{gather*}
    
	\item Se somete a un alumno a un examen de tipo Verdadero - Falso que contiene 10 preguntas; para que apruebe, debe responder correctamente a 8 preguntas o m�s. Si el estudiante est� adivinando,  �cu�l es la probabilidad de que apruebe el examen?
    \\\textbf{Soluci�n}
    
    \begin{gather*}
        P|8| = \binom{10}{8}(\frac{1}{2})^8(\frac{1}{2})^2 = \frac{45}{1024}\\
        P|9| = \binom{10}{9}(\frac{1}{2})^9(\frac{1}{2})^1 = \frac{10}{1024}\\
        P|9| = \binom{10}{10}(\frac{1}{2})^{10}(\frac{1}{2})^0 = \frac{1}{1024}\\
        P(8 \cup 9 \cup 10) = \frac{56}{1024} = 5.46\% 
    \end{gather*}
    
    \item En el curso de Plastilina II se distribuye un examen con 10 preguntas de opci�n m�ltiple. Para aprobarlo, se requiere responder correctamente a 7 o m�s de las preguntas. Si se supone que est� adivinando la respuesta en cada pregunta, �Cu�l es la probabilidad de aprobar el examen si las primeras 5 preguntas tienen 3 respuestas ocionales y las �ltimas 5 preguntas tienen 4 respuestas opcionales?
    \\\textbf{Soluci�n}\\
    7p:
    \begin{gather*}
         5p 2p = \binom{5}{5}(\frac{1}{3})^{5}(\frac{2}{3})^0 + \binom{5}{2}(\frac{1}{4})^{2}(\frac{3}{4})^3\\
         4p 3p = \binom{5}{4}(\frac{1}{3})^{4}(\frac{2}{3})^1 + \binom{5}{3}(\frac{1}{4})^{3}(\frac{3}{4})^2\\
         3p 4p = \binom{5}{3}(\frac{1}{3})^{3}(\frac{2}{3})^2 + \binom{5}{4}(\frac{1}{4})^{4}(\frac{3}{4})^1\\
         2p 5p = \binom{5}{2}(\frac{1}{3})^{2}(\frac{2}{3})^3 + \binom{5}{5}(\frac{1}{4})^{5}(\frac{3}{4})^0\\
    \end{gather*}
    
    8p:
    \begin{gather*}
         5p 3p = \binom{5}{5}(\frac{1}{3})^{5}(\frac{2}{3})^0 + \binom{3}{5}(\frac{1}{4})^{3}(\frac{3}{4})^2\\
         4p 4p = \binom{5}{4}(\frac{1}{3})^{4}(\frac{2}{3})^1 + \binom{4}{5}(\frac{1}{4})^{4}(\frac{3}{4})^1\\
         3p 5p = \binom{5}{3}(\frac{1}{3})^{3}(\frac{2}{3})^2 + \binom{5}{5}(\frac{1}{4})^{5}(\frac{3}{4})^0\\
    \end{gather*}
    
    9p:
    \begin{gather*}
         5p 4p = \binom{5}{5}(\frac{1}{3})^{5}(\frac{2}{3})^0 + \binom{5}{4}(\frac{1}{4})^{4}(\frac{3}{4})^1\\
         4p 5p = \binom{5}{4}(\frac{1}{3})^{4}(\frac{2}{3})^1 + \binom{5}{5}(\frac{1}{4})^{5}(\frac{3}{4})^0\\
    \end{gather*}
     
    10p:
    \begin{gather*}
         5p 5p = \binom{5}{5}(\frac{1}{3})^{5}(\frac{2}{3})^0 + \binom{5}{5}(\frac{1}{4})^{5}(\frac{3}{4})^0\\
    \end{gather*}
    
    \begin{gather*}
        P(x \leq 7) = \frac{23}{31104} = 7.39x10^{-4} � .074%\\
    \end{gather*}
    
    %Problema 10 lista 2 
    	10. En Chilpancingo, la incopatibilidad se da como la raz�n o motivo legal en el 70\% de todos los casos de divorcio. Obtenga la probabilidad de que 5 de los 6 siguientes divorcios archivados de esta ciudad argumentan incopatibilidad como motivo principal.
    \\\textbf{Soluci�n}\\La variable es binomial
    \begin{gather*}    
    	n = 6\\
    	p = .70 \\
    	q = .30 \\
    	P(5) = \binom{6}{5}(.7)^5(.3)^1 = .302526 � 30.35\%
    \end{gather*}
    
\end{enumerate}
\end{document}
