\documentclass{article}
\usepackage[utf8]{inputenc}
\inputencoding{utf8}
\usepackage{amsmath}
\usepackage{amsfonts}
\usepackage{amssymb}
\usepackage{graphicx}
\usepackage{listings}
\usepackage{ragged2e}
\usepackage{cancel}

\author{Hugue}
\title{\textbf{Parte 6 Lista Monti (Es de la lista 2 :v)}}

\begin{document}
\maketitle

\section*{}
16.-Se ha determinado que el número de camiones que llegan cada hora un almacén tiene la distribución que se muestra en la tabla. Calcule el numero esperado de llegadas por hora y la varianza de esta distribución.\\

\vspace{0.2cm}
\begin{tabular}{|c|c|c|c|c|c|c|c|}
\hline 
Número de Camiones & 0 & 1 & 2 & 3 & 4 & 5 & 6 \\ 
\hline 
Probabilidad & 0.05 & 0.10 & 0.15 & 0.25 & 0.30 & 0.10 & 0.05 \\ 
\hline 
\end{tabular}\\\\\\
Solución\\
La variable es discreta\\
i)
\begin{gather*}
	P(\Omega) = 1\\
	P(\Omega) = \Sigma f(x) = 0.05 + 0.10 + 0.5 + 0.25 +0.30 + 0.10 + 0.05 = 1\\
	\mu = \Sigma xf(x)\\
	\mu = 0(0.05) + 1(0.10) + 2(0.5) + 3(0.25) + 4(0.30) + 5(0.10) + 6(0.05)\\
	\mu = 3.15	
\end{gather*}
ii)
\begin{gather*}
	\sigma^2 = E[(x - \mu)^2] = E[x^2] - \mu ^2\\
	\sigma^2 = 0^2(0.05) + 1^2(0.10) + 2^2(0.5) + 3^2(0.25) + 4^2(0.30) + 5^2(0.10) + 6^2(0.05) - (3.15)^2\\
	\sigma^2 = 12.03 - 9.9225\\
	\sigma^2 = 2.1275
\end{gather*}\\\\\\
\section*{}
17.-En la siguiente tabla se identifica la probabilidad de que el sistema de computación se caiga el número señalado por periodos por semana, durante la fase inicial de instalación del sistema.\\
Calcule el número esperado de veces por semana que la computadora no está trabajando y la varianza de esta distribución\\
\begin{tabular}{|c|c|c|c|c|c|c|}
\hline 
Número de Periodos  & 4 & 5 & 6 & 7 & 8 & 9 \\ 
\hline 
Probabilidad & 0.01 & 0.08 & 0.29 & 0.42 & 0.14 & 0.06 \\ 
\hline 
\end{tabular} \\\\\\
Solución\\
La variable es discreta\\
i)
\begin{gather*}
	P(\Omega) = 1\\
	P(\Omega) = \Sigma f(x) = 0.01 + 0.08 + 0.29 + 0.42 + 0.14 + 0.06 = 1\\
	\mu = \Sigma xf(x)\\
	\mu = 4(0.01) + 5(0.08) + 6(0.29) + 7(0.42) + 8(0.14) + 9(0.06)\\
	\mu = 6.78	
\end{gather*}\\
ii)
\begin{gather*}
	\sigma^2 = E[(x - \mu)^2] = E[x^2] - \mu ^2\\
	\sigma^2 = 4^2(0.01) + 5^2(0.08) + 6^2(0.29) + 7^2(0.42) + 8^2(0.14) + 9^2(0.06) - (6.78)^2
	\sigma^2 = 47 - (6.78)^2
	\sigma^2 = 1.0376
\end{gather*}\\
18.- Obtenga el valor esperado de la siguiente función\\
\begin{gather*}
f(x)= \left\{ \begin{array}{lcc}
             \frac{1}{8}(x - 1) & 2 \leq x \leq 4 \\\\
             \\0  & otro \\
             \end{array}
   \right.
\end{gather*}\\
Solución\\
La variable es continua\\
\begin{center}
%P(\Omega) = \frac{1}{8}(\frac{x^2}{2} - x)\right.|_2^4\\
\begin{gather*}
	P(\Omega) = 1\\
	P(\Omega) = \int_{-\infty}^{\infty} f(x)dx = \int_{-\infty}^{2} f(x)dx + \int_{2}^{4} f(x)dx + \int_{4}^{\infty} f(x)dx\\
	P(\Omega) = 0 + \int_{2}^{4}\frac{1}{8}(x - 1) dx + 0\\
	P(\Omega) = \frac{37}{12}\\
	P(\Omega) = 3.0833\\
\end{gather*}\\
\end{center}
19.-Si $x$ es una variable aleatoria binomial, ¿Para qué valor de $p$ es la probabilidad binomial un máximo?\\
Solución\\
La variable es binomial\\
\begin{gather*}
	f(x) = {n \choose x}p^xq^{n-x}\\
	f'(x) =  {n \choose x}\dot[-(n-x)p^x(1-p)^{n-x-1} + xp^{x-1}(1 - p)^{n-x}]\\
\end{gather*}
Se iguala a cero para obtener el Máximo
\begin{gather*}
	\displaystyle\cancel{{n \choose x}}[-(n-x)p^x(1 - p)^{n-x-1} + xp^{x-1}(1 - p)^{n-x}] = 0\\
	xp^{x-1}(1 - p)^{n-x} = (n-x)p^x(1 - p)^{n-x-1}\\
	x(1 - p) = (n - x)p\\
	x - \cancel{xp} = np - \cancel{xp}\\
	p = \frac{x}{n}
\end{gather*}\\
20.- Demuestre que la media y la varianza de la distribución binomial son:\\
	$$ \mu = np\hspace{2cm} \sigma^2 = npq$$\\\\\\\\\\\\
Solución.\\
Sea $x$ una variable aleatoria binomial\\
i)
\begin{gather*}
	\mu = E[x] = \displaystyle\sum_{k=0}^N xf(x) = \displaystyle\sum_{k=0}^N x{n \choose x}p^xq^{n-x}\\
	\mu =  \displaystyle\sum_{k=0}^N x\frac{n!}{(n-x)!x!}p^xq^{n-x}\\
	\mu = \displaystyle\sum_{k=0}^N \frac{\cancel{x}n(n-1)!}{(n-x)!\cancel{x}(x-1)!}pp^{x-1}q^{n-x}\\
	\mu = np\displaystyle\sum_{k=0}^N \frac{(n-1)!}{(n-x)!(x-1)!}p^{x-1}q^{n-x}\\
	\mu = np\displaystyle\sum_{k=0}^N {n-1 \choose x - 1}p^{x-1}q^{n-x}\\
	\mu = np(1)\\
	\mu = np
\end{gather*}
ii)\
\begin{gather*}
	\sigma^2 = E[x^2] - \mu^2 = *E[x^2] - (np)^2\\
	*E[x^2] = E[x^2 + 0] = E[x^2 +x -x] = E[x^2 - x]+ E[x] = **E[x^2 - x]+ np\\
	**E[x^2 - x] = E[x(x-1)] = \displaystyle\sum_{k=0}^N x(x-1){n \choose x}p^xq^{n-x}\\
	**E[x^2 - x] = \displaystyle\sum_{k=2}^N x(x-1){n \choose x}p^xq^{n-x}\\
	**E[x^2 - x] = \displaystyle\sum_{k=2}^N\frac{x(x-1)!n!}{(n-x)!x!}p^xq^{n-x}\\
	**E[x^2 - x] = \displaystyle\sum_{k=2}^N\frac{\cancel{x}\cancel{(x-1)}n(n-1)(n-2)!}{(n-x)!\cancel{x}\cancel{(x-1)}(x-2)!}p^2p^{x-2}q^{n-x}\\
	**E[x^2 - x] = n(n-1)p^2 \displaystyle\sum_{k=2}^N \frac{(n-2)!}{(n-x)!(x-2)!}p^{x-2}q^{n-x}\\
	m = n -2\\
	y = x-2\\
	= n(n-1)p^2\displaystyle\sum_{y=0}^m \frac{m!}{(m-y)!y!}p^yq^{m-y}\\
	= n(n-1)p^2(1)\\
	= n(n-1)p^2\\
	\therefore E[x^2] = n(n-1)p^2 + np\\\\
	\sigma^2 = n(n-1)p^2 +np -(np)^2\\
	\sigma^2 = \cancel{n^2p^2} - np^2 +np - \cancel{n^2p^2}\\
	\sigma^2 = np(1-p)\\
	\sigma^2 = npq
\end{gather*}\\
21.- Demuestre que la media de la distribución geométrica está dada por:
$$ \mu = \frac{1}{p}$$\\
Solución\\
Sea x una variable aleatoria geométrica
\begin{gather*}
	\mu = \displaystyle\sum_{x=0}^n xf(x)\\
	\mu = \displaystyle\sum_{x=0}^n xpq^{x-1} = p \displaystyle\sum_{x=0}^n xq^{x-1}\\
	\mu = p \displaystyle\sum_{x=0}^n \frac{d}{dx}[q^x] = p \frac{d}{dx}[\displaystyle\sum_{x=0}^n q^x]\\
	 "\displaystyle\sum_{x=0}^n q^x = \frac{1}{1-q}"\\
	 \mu = p\frac{d}{dx}[\frac{1}{1-q}]\\
	 \mu = p[\frac{1}{(1-q)^2}]= \cancel{p}(\frac{1}{p^{\cancel{2}}})\\
	 \mu = \frac{1}{p}
\end{gather*}






\end{document}